
\let\negmedspace\undefined
\let\negthickspace\undefined
\documentclass[journal,12pt,twocolumn]{IEEEtran}
\usepackage{cite}
\usepackage{amsmath,amssymb,amsfonts,amsthm}
\usepackage{algorithmic}
\usepackage{graphicx}
\usepackage{textcomp}
\usepackage{xcolor}
\usepackage{txfonts}
\usepackage{listings}
\usepackage{enumitem}
\usepackage{mathtools}
\usepackage{gensymb}
\usepackage[breaklinks=true]{hyperref}
\usepackage{tkz-euclide} 
\usepackage{listings}
\usepackage{gvv}
%
%\usepackage{setspace}
%\usepackage{gensymb}
%\doublespacing
%\singlespacing

%\usepackage{graphicx}
%\usepackage{amssymb}
%\usepackage{relsize}
%\usepackage[cmex10]{amsmath}
%\usepackage{amsthm}
%\interdisplaylinepenalty=2500
%\savesymbol{iint}
%\usepackage{txfonts}
%\restoresymbol{TXF}{iint}
%\usepackage{wasysym}
%\usepackage{amsthm}
%\usepackage{iithtlc}
%\usepackage{mathrsfs}
%\usepackage{txfonts}
%\usepackage{stfloats}
%\usepackage{bm}
%\usepackage{cite}
%\usepackage{cases}
%\usepackage{subfig}
%\usepackage{xtab}
%\usepackage{longtable}
%\usepackage{multirow}
%\usepackage{algorithm}
%\usepackage{algpseudocode}
%\usepackage{enumitem}
%\usepackage{mathtools}
%\usepackage{tikz}
%\usepackage{circuitikz}
%\usepackage{verbatim}
%\usepackage{tfrupee}
%\usepackage{stmaryrd}
%\usetkzobj{all}
%    \usepackage{color}                                            %%
%    \usepackage{array}                                            %%
%    \usepackage{longtable}                                        %%
%    \usepackage{calc}                                             %%
%    \usepackage{multirow}                                         %%
%    \usepackage{hhline}                                           %%
%    \usepackage{ifthen}                                           %%
  %optionally (for landscape tables embedded in another document): %%
%    \usepackage{lscape}     
%\usepackage{multicol}
%\usepackage{chngcntr}
%\usepackage{enumerate}

%\usepackage{wasysym}
%\documentclass[conference]{IEEEtran}
%\IEEEoverridecommandlockouts
% The preceding line is only needed to identify funding in the first footnote. If that is unneeded, please comment it out.

\newtheorem{theorem}{Theorem}[section]
\newtheorem{problem}{Problem}
\newtheorem{proposition}{Proposition}[section]
\newtheorem{lemma}{Lemma}[section]
\newtheorem{corollary}[theorem]{Corollary}
\newtheorem{example}{Example}[section]
\newtheorem{definition}[problem]{Definition}

\newcommand{\BEQA}{\begin{eqnarray}}
\newcommand{\EEQA}{\end{eqnarray}}
\newcommand{\define}{\stackrel{\triangle}{=}}
\theoremstyle{remark}
\newtheorem{rem}{Remark}


\begin{document}


\bibliographystyle{IEEEtran}


\vspace{3cm}

\title{
	Question 1.5.9
}

\author{
	EE22BTECH11054 - Umair Parwez
}	

\maketitle
\newpage
\bigskip

\renewcommand{\thefigure}{\theenumi}
\renewcommand{\thetable}{\theenumi}

\begin{flushleft}
	Required to find points of contact, $E_{3}$ and $F_{3}$, of incircle with sides AC and AB respectively.
\end{flushleft}

\begin{flushleft}

	From previous questions we know the coordinates of the incircle are : \\
	\bigskip
	I = 
	\begin {bmatrix}
		\frac{-53-11\sqrt{37}+7\sqrt{61}+\sqrt{2257}}{12} \\ \\
		\frac{5-\sqrt{37}+5\sqrt{61}-\sqrt{2257}}{12}
	\end{bmatrix}
	\\
	\bigskip
	Radius of incircle is :\\
	\bigskip
	$r = \frac{185+41\sqrt{37}+-37\sqrt{61}-\sqrt{2257}}{6\sqrt{74}}$
	\\
	\bigskip
	Equation of incircle is : \\
	\bigskip
	${\lvert \lvert x-I \rvert \rvert}^2 = {r}^2$ -----(1)\\
	\bigskip

	points A, B and C are : \\
	\bigskip
	$A = \begin{bmatrix}
		1\\
		-1
	\end{bmatrix}, 
	B = \begin{bmatrix}
		-4\\
		6
	\end{bmatrix}, 
	C = \begin{bmatrix}
		-3\\
		-5
	\end{bmatrix}$
	\\
	\bigskip

	Parametric equation of AC is :\\
	\bigskip
	$x = A + k(A-B)$\\
	\bigskip
	$x = \begin{bmatrix}
		1\\
		-1
	\end{bmatrix}
	+ k\begin{bmatrix}
		4\\
		4
	\end{bmatrix}$\\
	\bigskip
	On simplification : \\
	\bigskip
	$x = \begin{bmatrix}
		1+k\\
		-1+k
	\end{bmatrix}$ -----(2)\\

	\bigskip
	On substituting Eq(2) in Eq(1) and solving, we find that k has only one solution :\\
	\bigskip
	$k = \frac{-4-\sqrt{37}+\sqrt{61}}{2}$\\
	\bigskip
	Substituting back into Eq(2), we get point of contact with AC,\\
	\bigskip
	${E}_3 = \begin{bmatrix}
		\frac{-2-\sqrt{37}+\sqrt{61}}{2}\\ \\
		\frac{-6-\sqrt{37}+\sqrt{61}}{2}
	\end{bmatrix}$\\
	\bigskip
	Parametric equation of AB is : \\
	\bigskip
	$x = A + k(A-B)$\\
	\bigskip
	$x = \begin{bmatrix}
		1\\
		-1
	\end{bmatrix}
	+ k\begin{bmatrix}
		5\\
		-7
	\end{bmatrix}$\\
	\bigskip
	On simplification :\\
	\bigskip
	$x = \begin{bmatrix}
		1+5k\\
		-1-7k
	\end{bmatrix}$\\
	\bigskip

	Substituting back into Eq(2), we get point of contact with AB,\\
	\bigskip
	${F}_3 = \begin{bmatrix}
		\frac{-111-20\sqrt{37}+5\sqrt{2257}}{74}\\ \\
		\frac{185+28\sqrt{37}-7\sqrt{2257}}{74}
	\end{bmatrix}$\\
	\bigskip
	Diagram is shown below :\\

\end{flushleft}

\begin{figure}[h]
	\centering
	\includegraphics[width=\columnwidth]{./Diagram.png}
	\caption{Points of contact of incircle}
	\label{Incircle}
\end{figure}

\end{document}


